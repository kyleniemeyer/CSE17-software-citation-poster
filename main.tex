\PassOptionsToPackage{usenames,dvipsnames}{xcolor}

\documentclass[20pt, a0paper, landscape, margin=0mm, innermargin=15mm, blockverticalspace=15mm, colspace=15mm, subcolspace=8mm]{tikzposter}
\usepackage[T1]{fontenc}
\usepackage[utf8]{inputenc}
\usepackage[english]{babel}

\usepackage{amsmath,amsfonts,amssymb,mathtools}

%\usepackage[usenames,dvipsnames]{xcolor}
\usepackage{graphicx,mwe}
\usepackage{filecontents}% http://ctan.org/pkg/filecontents
\usepackage{tikz}
\usepackage{multicol}
\usepackage{adjustbox}
\usepackage{authblk}

\usepackage{enumitem}

\usepackage{caption}
\captionsetup{font=large}

\usepackage{url}
\usepackage[colorlinks=false]{hyperref}
\urlstyle{tt}


%\newcommand*{\doi}[1]{\href{https://doi.org/#1}{\nolinkurl{https://doi.org/#1}}}
\newcommand*{\doi}[1]{\href{https://doi.org/#1}{\nolinkurl{doi:#1}}}


% university colors based on branding guides
\definecolor{OSUorange}{HTML}{C34500}
\definecolor{UConnBlue}{HTML}{000E2F}
\definecolor{GoogleBlue}{HTML}{0266C8}

% \makeatletter
% \def\TP@titlegraphictotitledistance{-4cm}
% \settitle{ \centering \vbox{
% \@titlegraphic \\ [\TP@titlegraphictotitledistance] 
% \centering
% \color{titlefgcolor} {\bfseries \Huge \@title \par} % add \sc for smallcaps
% \vspace*{2em}
% {\huge \@author \par} \vspace*{1em} {\LARGE \@institute}
% }}
% \makeatother

%% Code for increasing tikzfigure caption font size
% \renewenvironment{tikzfigure}[1][]{
%   \def \rememberparameter{#1}
%   \vspace{10pt}
%   \refstepcounter{figurecounter}
%   \centering
%   }{
%     \ifx\rememberparameter\@empty
%     \else %nothing
%     \\[10pt]
%     {\large Fig.~\thefigurecounter: \rememberparameter}
%     \fi
% }


%% Set up a logo on each side
\makeatletter
\newcommand\insertlogoi[2][]{\def\@insertlogoi{\includegraphics[#1]{#2}}}
\newcommand\insertlogoii[2][]{\def\@insertlogoii{\includegraphics[#1]{#2}}}
\newlength\LogoSep
\setlength\LogoSep{-2cm}

\insertlogoi[width=18cm]{OSU-color-horz}
\insertlogoii[width=18cm]{UConn-color-horz}

\def\TP@titlegraphictotitledistance{-4cm}
\settitle{ \centering \vbox{
%\@titlegraphic \\ [\TP@titlegraphictotitledistance] 
\centering
\color{titlefgcolor} {\bfseries \Huge \@title \par} % add \sc for smallcaps
\vspace*{2em}
{\huge \@author \par} \vspace*{1em} {\LARGE \@institute}
}}

\renewcommand\maketitle[1][]{  % #1 keys
    \normalsize
    \setkeys{title}{#1}
    % Title dummy to get title height
    \node[transparent,inner sep=\TP@titleinnersep, line width=\TP@titlelinewidth, anchor=north, minimum width=\TP@visibletextwidth-2\TP@titleinnersep]
        (TP@title) at ($(0, 0.5\textheight-\TP@titletotopverticalspace)$) {\parbox{\TP@titlewidth-2\TP@titleinnersep}{\TP@maketitle}};
    \draw let \p1 = ($(TP@title.north)-(TP@title.south)$) in node {
        \setlength{\TP@titleheight}{\y1}
        \setlength{\titleheight}{\y1}
        \global\TP@titleheight=\TP@titleheight
        \global\titleheight=\titleheight
    };

    % Compute title position
    \setlength{\titleposleft}{-0.5\titlewidth}
    \setlength{\titleposright}{\titleposleft+\titlewidth}
    \setlength{\titlepostop}{0.5\textheight-\TP@titletotopverticalspace}
    \setlength{\titleposbottom}{\titlepostop-\titleheight}

    % Title style (background)
    \TP@titlestyle

    % Title node
    \node[inner sep=\TP@titleinnersep, line width=\TP@titlelinewidth, anchor=north, minimum width=\TP@visibletextwidth-2\TP@titleinnersep]
        at (0,0.5\textheight-\TP@titletotopverticalspace)
        (title)
        {\parbox{\TP@titlewidth-2\TP@titleinnersep}{\TP@maketitle}};

    \node[inner sep=0pt,anchor=west] 
      at ([xshift=-\LogoSep]title.west)
      {\@insertlogoi};

    \node[inner sep=0pt,anchor=east] 
      at ([xshift=\LogoSep]title.east)
      {\@insertlogoii};

    % Settings for blocks
    \normalsize
    \setlength{\TP@blocktop}{\titleposbottom-\TP@titletoblockverticalspace}
}
\makeatother

\setlength{\columnsep}{2cm}


\title{\parbox{0.7\linewidth}{\centering Minisymposterium: The Journal of Open Source Software}}

\author[1]{Arfon M.\ Smith}
% \author[2]{Lorena A.\ Barba}
% \author[3]{George Githinji}
% \author[4]{Melissa Gymrek}
% \author[5]{Kathryn Huff}
% \author[6]{Daniel S.\ Katz}
% \author[7]{Christopher R.\ Madan}
% \author[8]{Abigail Cabunoc Mayes}
% \author[9]{Kevin M.\ Moerman}
% \author[10]{Kyle E.\ Niemeyer}
% \author[11]{Pjotr Prins}
% \author[12]{Karthik Ram}
% \author[13]{Tracy K.\ Teal}
% \author[14]{Jake Vanderplas}

\affil[1]{School of Mechanical, Industrial, and Manufacturing Engineering, Oregon State University}

%\affil[10]{School of Mechanical, Industrial, and Manufacturing Engineering, Oregon State University}

\institute{\href{mailto:kyle.niemeyer@oregonstate.edu}{\nolinkurl{kyle.niemeyer@oregonstate.edu}}}

\date{}



\usetitlestyle{Filled}

\begin{document}

\maketitle

\begin{columns} % See Section 4.4
    \column{0.25} % See Section 4.4
    \block{Introduction \& Motivation}{
    This project is developing {\fontfamily{pag}\selectfont \textbf{SLACKHA}}: a software library for accelerating the computation of chemical kinetics terms in simulations of reactive fluid flows, optimized for hybrid CPU\slash GPU processing architectures.
    \vspace{0.75em}
    
    The motivations behind the project are two-pronged:
    \begin{itemize}
        \item The prohibitive computational cost of incorporating detailed, accurate models of chemical kinetics, and 
        \item The trend of high-performance computing clusters towards a hybrid processing paradigm using massively parallel accelerators like GPUs.
    \end{itemize}
    \vspace{0.5em}
    Reactive-flow simulations pose high computational expenses when using detailed chemical kinetic models, primarily due to
    \begin{enumerate}
        \item The large---and ever-increasing---model sizes, depicted in Fig.~\ref{fig:sizes}, and
        \item Stiffness encountered when integrating the ordinary differential equations for chemical kinetics, as shown in Fig.~\ref{fig:stiffness}.
    \end{enumerate}
    
    % \begin{minipage}{0.48\linewidth}
    % \centering
    % \begin{tikzfigure}
    % \includegraphics[width=\linewidth]{mechanism-survey}
    % \end{tikzfigure}
    % \captionof{figure}{Sizes of detailed kinetic models for hydrocarbon oxidation.}\label{fig:sizes}
    % \end{minipage}
    % \hfill
    % \begin{minipage}{0.49\linewidth}
    % \centering
    % \begin{tikzfigure}
    % \includegraphics[width=\linewidth]{species-timescales}
    % \end{tikzfigure}
    % \vspace{0.25em}
    % \captionof{figure}{Characteristic species creation timescales for CH$_4$ oxidation.}\label{fig:stiffness}
    % \end{minipage}
    
    \vspace{1em}
    
    {\fontfamily{pag}\selectfont SLACKHA} will consist of components that can cluster chemically similar cells and detect local stiffness of chemical kinetics governing equations and adaptively select the most efficient integrator based on available hardware.
    Improved computational performance will be demonstrated by coupling with commercial and open-source computational fluid dynamics software.
    }
    
    \column{0.5}
    \block{Components of {\fontfamily{pag}\selectfont SLACKHA}}{
    
    % \begin{tikzfigure}
    % \includegraphics[width=0.75\linewidth]{SLACKHA-components}
    % \end{tikzfigure}
    % \captionof{figure}{Functionality of {\fontfamily{pag}\selectfont SLACKHA} with all components and connections.}
    
    }
    
    \column{0.25}
    \block{Project Plan}{
    The three-year project plan for the development {\fontfamily{pag}\selectfont SLACKHA} is a collaborative effort between Oregon State University and the University of Connecticut, with professional software development contributions from Dr.~Christopher Stone of Computational Science \& Engineering.
    
    % \begin{tikzfigure}
    % \includegraphics[width=\linewidth]{NSF-SI2-project-timeline}
    % \end{tikzfigure}
    % \captionof{figure}{Three-year plan for {\fontfamily{pag}\selectfont SLACKHA} development.}
    
    \vspace{2em}
    
    Although case studies will be performed using combustion phenomena, other reacting-flow communities will be engaged, including astrophysics, atmospheric chemistry, nuclear reactions, and biochemical networks.
    }
\end{columns}

% bottom 
\begin{columns}
    \column{0.25}
    \block{Basis}{
    Our approach to accelerating chemical kinetics relies on the operator-splitting or fractional-step technique to decouple the chemistry terms from the transport terms and other physical processes. This results in a large system of independent ODEs that need to be integrated each time step:
    
    % \begin{tikzfigure}
    % \includegraphics[width=\linewidth]{chemistry-decoupling}
    % \end{tikzfigure}
    % \captionof{figure}{Depiction of decoupling of chemical kinetics ODEs in reactive-flow simulations using operator splitting.\textsuperscript{a}}
    }

    \column{0.25}
    \block{Products}{
    \begin{enumerate}[leftmargin=1cm,topsep=0pt]
        
        \item NJ Curtis \& KE Niemeyer. \texttt{accelerInt}. GitHub: \url{https://github.com/SLACKHA/accelerInt}.
        v1.0 (2017): \doi{10.5281/zenodo.230256}
        
        \item KE Niemeyer \& NJ Curtis. \texttt{pyJac}. GitHub: \url{https://github.com/SLACKHA/pyJac}.
        v1.0.2 (2017): \doi{10.5281/zenodo.251144}
        
        %\item Andrew Alferman \& Kyle E.\ Niemeyer. ``Investigation of stiffness detection metrics for chemical kinetics.'' In preparation.
        
        \item CP Stone, AT Alferman, \& KE Niemeyer.
        ``Accelerating finite-rate chemical kinetics with coprocessors: comparing vectorization methods on GPUs, MICs, and CPUs.'' (2017) In preparation, available via {\tt \href{http://arxiv.org/abs/1608.05794}{arXiv:1608.05794} [physics.comp-ph]}.
        
        \item KE Niemeyer, NJ Curtis, \& CJ Sung. ``\texttt{pyJac}: analytical Jacobian generator for chemical kinetics.'' (2017) \textit{Computer Physics Communications}, in press. \doi{10.1016/j.cpc.2017.02.004} 
        
        \item NJ Curtis, KE Niemeyer, \& CJ Sung. ``An investigation of GPU-based stiff chemical kinetics integration methods.'' (2017) \textit{Combustion \& Flame}, in press. \doi{10.1016/j.combustflame.2017.02.005}
        %
        %\item Kyle E.\ Niemeyer, Nicholas J.\ Curtis, and Chih-Jen Sung. ``Initial investigation of \texttt{pyJac}: an analytical Jacobian generator for chemical kinetics.'' Fall 2015 Meeting of the Western States Section of the Combustion Institute, Provo, UT, USA. 5--6 October 2015. \href{https://dx.doi.org/10.6084/m9.figshare.2075515.v1}{\nolinkurl{doi:10.6084/m9.figshare.2075515.v1}}
    \end{enumerate}
    %\vspace{0.1em}
    }
    
    \column{0.25}
    \block{Acknowledgements}{
    This work is supported by NSF grants ACI-1535065 and ACI-1534688. We also acknowledge the efforts of graduate research assistants Nicholas Curtis (UConn) and Andrew Alferman (Oregon State), and undergraduate research assistant Parker Clayton (Oregon State).
    \\ \\
    \begin{minipage}{0.42\linewidth}
    \begin{tikzfigure}
    \includegraphics[width=\linewidth]{cc-by}
    \end{tikzfigure}
    \end{minipage}
    \hfill
    \begin{minipage}{0.55\linewidth}
    This work is licensed under a Creative Commons Attribution 4.0 International License. To view a copy of this license, visit \url{http://creativecommons.org/licenses/by/4.0/}.
    \end{minipage}
    
    \vspace{2.5em}
    \innerblock{Image credits}{
    \small{
    a: Modified from \url{https://www.olcf.ornl.gov/wp-content/uploads/2011/06/Oefelein2_web.png}, 
    b: Weber et al., \emph{Combust Flame} (2014) \doi{10.1016/j.combustflame.2014.01.018}
    c: Intel Skylake CPU \url{http://wimages.vr-zone.net/2015/04/Intel.jpg}
    d: NVIDIA Tesla K40 \url{http://www.wiredzone.com/mmenglish/Images/actual}
    e: Intel Xeon Phi \url{https://streamcomputing.eu/wp-content/uploads/2012/11/intel-xeon-phi-card.jpg}
    }}
    }
    
    \column{0.25}
    \block{Further Information}{
    {\colorlet{innerblocktitlebgcolor}{OSUorange}
    \colorlet{innerblocktitlefgcolor}{white}
    \innerblock{Niemeyer Research Group (OSU)}{\centering{\url{http://kyleniemeyer.com}}}
    }
    {\colorlet{innerblocktitlebgcolor}{UConnBlue}
    \colorlet{innerblocktitlefgcolor}{white}
    \innerblock{Combustion Diagnostics Laboratory (UConn)}{\centering{\url{http://combdiaglab.engr.uconn.edu}}}
    }
    
    \vspace{2.5em}
    
    \begin{minipage}{\linewidth}
    \centering
    {\fontfamily{pag}\selectfont 
    \Huge
    \textbf{SLACKHA}
    }
    \end{minipage}
    \vspace{1em}
    
    {\colorlet{innerblocktitlebgcolor}{GoogleBlue}
    \colorlet{innerblocktitlefgcolor}{white}
    \innerblock{Project discussion forum}{\centering{\url{https://groups.google.com/d/forum/slackha-users}}}
    }
    \innerblock{Project homepage}{\centering{\url{http://slackha.github.io/}}}
    }
    
\end{columns}


\end{document}
